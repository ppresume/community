\documentclass[a4paper, serif, 12pt]{article}

\usepackage[utf8]{inputenc}

\usepackage[left=0.79in, right=0.79in, top=0.79in, right=0.79in]{geometry}
\usepackage[UTF8, fontset=none, heading=false, punct=kaiming, scheme=plain, space=auto]{ctex}

\setCJKmainfont{Noto Serif CJK SC}

\begin{document}
\setlength{\parskip}{1em}

\noindent
The Knuth–Plass algorithm is a line-breaking algorithm designed for use in
Donald Knuth's typesetting program TeX. It integrates the problems of text
justification and hyphenation into a single algorithm by using a discrete
dynamic programming method to minimize a loss function that attempts to
quantify the aesthetic qualities desired in the finished output.

\noindent
Making hyphenation decisions follows naturally from the algorithm, but the
choice of possible hyphenation points within words, and optionally their
preference weighting, must be performed first, and that information inserted
into the text stream in advance. Knuth and Plass' original algorithm does not
include page breaking, but may be modified to interface with a pagination
algorithm, such as the algorithm designed by Plass in his PhD thesis.
\end{document}
